\documentclass{memoir}
\usepackage{chessboard}
\usepackage{skak}
\usepackage{wasysym}
\usepackage[LSB,LSBC3,LSBC4,T1]{fontenc}
\usepackage[utf8]{inputenc}
\usepackage{url}

\tightlists

\title{How to play Tafl}
\newcommand{\subtitle}{Rediscovering Tablut, Hnefatafl, Brandubh and other variations of the Viking game}
\author{Jon Ericson}
\renewcommand{\maketitlehookb}{\centering\textsc{\subtitle}}

\makeatletter
\cbDefineNewPiece{white}{C}
  {\raisebox{\depth*3}{\cfss@whitepiececolor
    $\Circle$}}
  {\BlackEmptySquare%
   \makebox[0pt][r]{\cfss@whitepiececolor
    \raisebox{\depth*3}{%
     \makebox[1em]{$\Circle$}}}}
\cbDefineNewPiece{black}{c}
  {\raisebox{\depth*3}{\cfss@blackpiececolor
    $\CIRCLE$}}
  {\BlackEmptySquare%
   \makebox[0pt][r]{\cfss@blackpiececolor
    \raisebox{\depth*3}{%
      \makebox[1em]{$\CIRCLE$}}}}
\cbDefineNewPiece{white}{+}
  {\raisebox{\depth*3}{\cfss@blackpiececolor
    $\oplus$}}
  {\BlackEmptySquare%
   \makebox[0pt][r]{\cfss@blackpiececolor
    \raisebox{\depth*3}{%
     \makebox[1em]{$\oplus$}}}}
  \makeatother
  
  \def\tablutcorners{a1,a9,i1,i9}
  \def\tablutthrone{e5}
  \def\tablutcamps{a4,a5,a6,b5,d1,e1,f1,e2,d9,e9,f9,e8,i4,i5,i6,h5}
  

  \storechessboardstyle{tablut}{
    color=black,
    maxfield=i9,
    blackfieldcolor=white!100,  % https://tex.stackexchange.com/questions/200258/uncheckered-chess-board-using-the-chessboard-package/200270#200270
    setfontcolors,
    pgfstyle=border,
    linewidth=0.5pt,
    markboard,
    linewidth=1pt,
    markstyle=cross,
    markfields=\tablutcorners,  
    color=gray!50,
    backstyle=color,    
    backfields=\tablutthrone,
    color=gray!20,
    backstyle=color,    
    backfields=\tablutcamps,
  }

  \def\brandubhcorners{a1,a7,g1,g7}
  \def\brandubhthrone{d4}
  
  \storechessboardstyle{brandubh}{
    color=black,
    maxfield=g7,
    blackfieldcolor=white!100,
    setfontcolors,
    pgfstyle=border,
    linewidth=0.5pt,
    markboard,
    linewidth=1pt,
    markstyle=cross,    
    markfields=\brandubhcorners,
    color=gray!50,
    backstyle=color,    
    backfields=\brandubhthrone,
  }

  \def\puzzlecorners{a1,a5,e1,e5}
  \def\puzzlethrone{c3}

  \storechessboardstyle{puzzle}{
    color=black,
    maxfield=e5,
    blackfieldcolor=white!100,
    setfontcolors,
    pgfstyle=border,
    linewidth=0.5pt,
    markboard,
    linewidth=1pt,
    markstyle=cross,    
    markfields=\puzzlecorners,
    color=gray!50,
    backstyle=color,    
    backfields=\puzzlethrone,
  }

  \storechessboardstyle{key}{
    color=black,
    maxfield=h1,
    blackfieldcolor=white!100,
    setfontcolors,
    pgfstyle=border,
    linewidth=0.5pt,
    markboard,
    linewidth=1pt,
    markstyle=cross,    
    markfields={a1},
    color=gray!50,
    backstyle=color,    
    backfields={b1},
    addfen={2+Cc},
    color=gray!20,
    backstyle=color,    
    backfields={g1},
    color=black,
    markstyle=color,    
    markfields={h1},
  }
  
\newcommand{\diagramname}{Diagram}
\newcommand{\listdiagramname}{List of Diagrams}
\newlistof{listofdiagrams}{dgm}{\listdiagramname}
\newfloat{diagram}{dgm}{\diagramname}
\newlistentry{diagram}{dgm}{0}


\newtheorem{xexample}{Example}
\newcommand\listexamplesname{List of Examples}
\newlistof{listofexamples}{loe}{\listexamplesname}
\newlistentry[chapter]{examplecounter}{loe}{0}
\newcommand{\edescription}[1]{%
\refstepcounter{examplecounter}
\par\noindent\textbf{Example \theexamplecounter---#1.}
\addcontentsline{loe}{figure}{\protect\numberline{\theexamplecounter}#1}
}
\newenvironment{example}[1][]{\edescription{#1}\par\noindent}{}

\usepackage{answers}
\Newassociation{sol}{Solution}{ans}
\renewcommand{\Solutionlabel}[1]{\small{#1.}}
\newtheorem{ex}{}[section]
\renewcommand{\theex}{\arabic{ex}}
\newenvironment{Ex}[1]{\begin{trivlist}\item \textsc{#1} %
\renewcommand{\Currentlabel}{#1}}{\end{trivlist}}


\begin{document}
%\maketitle

\chapter{How Tafl is played}

A few years ago my wife bought me a really lovely Tablut board printed
on leather that could be cinched up into a sack for holding the
pieces. I absolutely love the way it looks. It fills my head with
romantic notions of playing out a tense match on a barrel head or
somesuch.  But when I played the game itself\dots well, it was
frustrating. From the first move to the end of the game, I could not
grok the strategy required. What's worse, I didn't see how I could
improve.

At last I did something I normally do by instinct: I searched the
internet for ``Tablut strategy''. What I discovered was that Tablut
was part of a long and nearly forgotten tradition of games that come,
not from a single mind, but was the result of generation after
generation refining a simple set of rules. 

The names given to these games seem nearly fantastical:

\begin{itemize}
\item Hnefatafl
\item Large Hnefatafl 
\item Alea Evangelii
\item Tablut
\item Ard Ri
\item Brandubh
\item Fidhchell
\item Gwyddbwyll
\item Tawlbwrdd
\end{itemize}



\section{Why the name ``Viking Chess'' is misleading}

This book isn't about the history of the Tafl family of games. There
are many excellent sources which I've included in an appendix. Instead
this is a book about how to excel when playing these ancient
games. But in order to accomplish that goal, we need to clear out some
assumptions that modern people will inevitably make from our habit of
calling it ``Viking Chess''.

Chess originated in the Indian subcontinent and traveled to Southern
Europe via Persia. From there it spread north to Scandinavia and the
rest of the world. Along the way, it was modified and refined until it
became the modern game we know.

Meanwhile, people in Northern Europe were already playing games that
likely descended from the ancient Roman game of \emph{Ludus
Latrunculorum}. Archaeologists have discovered boards carved on wood
(often with Nine Men's Morris on the reverse) or scratched onto
stone. These boards ranged in size from 7x7 to 18x18. Some were marked
to be played in the squares (as with Chess), others in the
intersections of lines (like Go) and still others used pegs.

While we have a good deal of evidence about the equipment used, the
rules of Tafl games have been largely lost to time. Before near
universal literacy, knowledge primarily passed from person to person
via the spoken word. Over time, Chess supplanted Tafl games in
Northern Europe, so the rules of the older game ceased to be
transmitted. Many games have been lost to us for this reason with only
tentative reconstructions created from cryptic clues left in ancient
writings.

Thankfully, Carl Linnaeus, who formalized modern biological taxonomy,
visited Lapland in 1732. Along the way, he observed the culture of the
Sámi people, including the games they played. One of the games, which
Linnaeus called Tablut, is played on a 9x9 board with a king, his 8
defenders and 16 attackers controlled by the other player
(diagram~\ref{tablut}). The rules he described have been instrumental
in reconstructing other Tafl games described in less detail by other
sources.

\begin{diagram}
  \centering
  \chessboard[
    style=tablut,
    addfen={3ccc/4c/4C/c3C3c/ccCC+CCcc/c3C3c/4C/4c/3ccc},
    mover=b,
  ]
  \caption{The initial position of a game of Tablut.}\label{tablut}
  The king ($\oplus$) rests on the central throne. His defenders
  ($\Circle$) extend out in the form of a cross. Attackers ($\CIRCLE$)
  begin in 4 camps at the edge of the board. Marked with an X are the
  four corners the king may use to exit the board and win the game.
\end{diagram}


It's not hard to see how this rediscovered game began to be called
``Viking Chess''. Both games play on a grid and exude a theme of
military strategy. Board Game Geek, the internet's largest database of
board games, lists both as ``Abstract Strategy'' with a ``Grid
Movement'' mechanic. And yet, despite his life's work, Linnaeus didn't
place the Sámi game in any taxonomy, much less label it as a variant
of Chess.

\begin{quote}
The game called \emph{Tablut} is played with a checkered board, and
twenty-five pieces, or men, in the following manner.---''July 21''
entry in \emph{A Tour in Lapland} by Carl von Linné and translated by
Charles Troilius
\end{quote}

If you want to play a Tablut (or any other Tafl game), treating it as
a variant of Chess produces disaster. In Chess, opponents eliminate
enemy forces to open up an attack on the king. In Tafl, one side hopes
to escort the king to safety while the other endeavors to trap him in
their net. The first time I won a game of Tablut
(section~\ref{firstwin}), I was shocked to see three of my pieces had
been captured and that I'd captured none of theirs. Indeed, while they
were taking my pieces in one corner, my king executed his escape to
the opposite corner.

Tafl games work on fundamentally different principles than
Chess. Experience with Go can be somewhat more useful in that the
attacker must encircle the king while the defender must prevent that
fate. Even so, the dynamic nature of the pieces mean the board state
can change dramatically from one turn to the next. It pays to play
Tafl games in the Tafl mindset.


\section{A note on notation}

\begin{diagram}
  \centering
  \chessboard[
    style=key,
  ]
  \caption{Key to Tafl diagrams}\label{key}
  \begin{enumerate}
  \item \textbf{a1} contains the exit square marked with an X. Only the
    king may enter the square.
  \item \textbf{b1} is shaded to indicate the king's throne.
  \item The king is standing on \textbf{c1}.
  \item One of the king's defenders is located on \textbf{d1}. 
  \item An attacker occupies \textbf{e1}. 
  \item \textbf{f1} is an ordinary square. 
  \item The attacker's camp is marked on \textbf{g1}. 
  \item Some puzzles use blocked squares as shown in \textbf{h1}. This
    isn't a normal part of Tafl boards.
  \item The right side of the board sometimes holds a colored square
    to indicate the side that must move next. 
  \end{enumerate}
\end{diagram}

Diagrams should be fairly easy to read once you know the game. I've
included a key in diagram~\ref{key}.

In addition to diagrams, I'm using a variation of Chess algebraic
notation that seems to be the common way to record Tafl games. Rows
are numbered from south to north starting with 1 and columns are
assigned letters starting with a on the east. Therefore a single
square can be identified with a letter and a number. For instance, the
bottom left square is called a1. Movement is show by a pair of squares
separated by a dash. So moving from the bottom right\footnote{Though
it is impossible for any piece other than the king to enter that
square and he never leaves it as the game will have finished at that
point.} to the next square to the left is recorded a1-b1.

Capturing is indicated by adding an x and the square that was
captured. So c2-c1xb1 means the piece on c2 captured piece on
b1.\footnote{We don't need to indicate the other piece that
participated in the capture since it's implied. In this case, it was
the a1 corner, not another piece.} On occasion a move results in more
than one capture. Further captured pieces will be indicated with a
/. For instance: c2-c1xb1/d1 mean both b1 and d1 were captured.

I'm avoiding naming the sides as ``black and white'' because there's a
better way to identify them: attacker and defender.\footnote{I also
use Swedes and Muscovites for the two sides when describing Tablut
following Linnaeus. It strongly suggests the sides acquired local
names based on the politics of the area. It's not at all hard to come
up with novel themes such as Celebrity vs. Paparazzi.} In Chess the
sides are symmetrical except for the first player. Tafl is inherently
asymmetrical, so for clarity it's better to identify the goals of each
side.

Generally the attackers make the first move, but not always. Whichever
side makes the first move is listed first in game logs. 

Other notations borrowed from Chess:

\begin{description}
\item[!]an interesting move
\item[!!]a brilliant move
\item[?]a bad move
\item[??]a blunder
\end{description}

\section{The goal of Tafl}

Whether you are playing Tablut, Hnefatafl, Brandubh or any of the
other dozen variations of Tafl, the goal remains the same:
\begin{itemize}
\item The king attempts to escape the board.
\item The attackers must stop the king from escaping.
\end{itemize}

Everything else is a distraction.


\section{Default rules}



\Opensolutionfile{ans}[king-escape]
\section{King escape puzzles}

One thing we can learn from Chess is the value of puzzles for learning
to play and get better at the game. For this book, I'm introducing a
5x5 board (diagram~\ref{puzzleblank}) that's probably too small for a
competitive Tafl game. But it's just right for Tafl puzzles. As
always, it helps to set up a physical board to play out the
puzzles. It's easiest to use a corner of a chessboard and any random
pieces or markers you have handy. Or you can print out or copy the
board this book on page~\pageref{puzzleblank}.



\begin{diagram}
  \centering
  \chessboard[
    style=puzzle,
    mover=w,
    largeboard,
  ]
  \caption{5x5 version of Tafl for learning the
    basics.}\label{puzzleblank}
  This board is too small for a competitive game, but is
  ideal for creating Tafl puzzles.
\end{diagram}


\begin{ex}
    \chessboard[
    style=puzzle,
    addfen={//2+},
    mover=w,
    tinyboard,
   ]
   
   \noindent How many moves must the king make to exit from a corner?

   \begin{sol}
     \chessboard[
       style=puzzle,
       addfen={//2+},
       mover=w,
       tinyboard,
       color=black,
       pgfstyle=straightmove,
       markmove=c3-a3,
       markmove=a3-a1
     ]
     
     It's just two moves. Since the board is symmetrical, there are
     essentially 4 equivalent ways to do the initial move. After
     you've broken the symmetry by moving once, there are two ways to
     end the game: a3-a1 or a3-5.
   \end{sol}
\end{ex}

\begin{ex}
   \chessboard[
    style=puzzle,
    addfen={//2+},
    color=black,
    markstyle=color,
    markfields={a3,c1,c5,e3},
    mover=w,
    tinyboard,
   ]
   
   \noindent The blacked out squares are impassible. How many moves
   must the king make to exit from a corner?

   \begin{sol}
     \chessboard[
       style=puzzle,
       addfen={//2+},
       mover=w,
       tinyboard,
       color=black,
       markstyle=color,
       markfields={a3,c1,c5,e3},
       pgfstyle=straightmove,
       markmove=c3-b3,
       markmove=b3-b1,
       markmove=b1-a1,
     ]
     
     The blocked squares add a move to the kings escape
     plans. Attackers won't let your king just leave, so it's common
     for the king to make a winding path to the exit.
   \end{sol}
\end{ex}

\begin{ex}
   \chessboard[
    style=puzzle,
    addfen={/2C/1C+C/2C},
    color=black,
    markstyle=color,
    markfields={a3,c1,c5,e3},
    mover=w,
    tinyboard,
   ]
   
   \noindent What if you have guards?

   \begin{sol}
     \chessboard[
       style=puzzle,
       addfen={/1CC/2+C/2C/},
       mover=w,
       tinyboard,
       color=black,
       markstyle=color,
       markfields={a3,c1,c5,e3},
       pgfstyle=straightmove,
       markmove=b3-b4,
       markmove=c3-b3,
       markmove=b3-b1,
       markmove=b1-a1,
     ]
     
     Notice that your own pieces get in the king's way. Moving the
     guard on b3 slows the king down by one move, so it takes 4 in
     total.  There are certainly times when you'll wish you have just
     one fewer pieces on the board.
   \end{sol}
\end{ex}

\begin{ex}
   \chessboard[
    style=puzzle,
    addfen={2c//c1+1c//2c},
    mover=w,
    tinyboard,
   ]
   
   \noindent Can the king escape if there are 4 attackers who can move in response?

   \begin{sol}
     Yes, unless captures are allowed.
     
     \chessboard[
       style=puzzle,
       addfen={2c//c+2c//1c},
       showmover=false,
       tinyboard,
       color=black,
       pgfstyle=straightmove,
       markmove=c3-b3,
       markmove=c1-b1,
     ]
     \chessboard[
       style=puzzle,
       addfen={2c//c+2c//1c},
       showmover=false,
       tinyboard,
       color=black,
       pgfstyle=straightmove,
       markmove=b3-b5,
       markmove=b5-a5,
     ]


     If it is possible to capture the king against the edge of the
     board and the exit (as is the case with many Tafl games), the
     king will be captured by 2\dots~b1-b4xb5. 
     
     \chessboard[
       style=puzzle,
       addfen={1+c/1c/c3c/},
       showmover=false,
       tinyboard,
       color=black,
       pgfstyle=straightmove,
       markmove=b1-b4,
     ]

     The attackers can continually lay that trap by carefully chasing
     the king around the board.

     \chessboard[
       style=puzzle,
       addfen={2c//c+2c//1c},
       showmover=false,
       tinyboard,
       color=black,
       pgfstyle=straightmove,
       markmove=c3-b3,
       markmove=c1-b1,
     ]
     \chessboard[
       style=puzzle,
       addfen={2c//c/1+2c/1c},
       showmover=false,
       tinyboard,
       color=black,
       pgfstyle=straightmove,
       markmove=b3-b2,
       markmove=e3-e2,
     ]
     \chessboard[
       style=puzzle,
       addfen={2c//c/3+c/3c},
       showmover=false,
       tinyboard,
       color=black,
       pgfstyle=straightmove,
       markmove=b2-d2,
       markmove=b1-d1,
     ]
     \chessboard[
       style=puzzle,
       addfen={2c/3+c/c//3c},
       showmover=false,
       tinyboard,
       color=black,
       pgfstyle=straightmove,
       markmove=d2-d4,
       markmove=e2-e4,
     ]

     
   \end{sol}
\end{ex}

\begin{ex}
   \chessboard[
    style=puzzle,
    addfen={1c/c2c/2+/4c},
    mover=w,
    tinyboard,
   ]
   
   \noindent The king can guarantee his escape. What is his next move?

   \begin{sol}
     According to Linnaeus, when a king has a path to escape, the
     defender must say \emph{raichi}. In this case, moving to a3, c5
     or e3 would give the king one way to escape. But each of those
     paths can be blocked by an attacker (e2-a2, d4-d5 or d4-e4). But
     if the king moves to c1 the defender would need to say
     \emph{tuichu}, which means he has two exits and has won the
     game. Both of those exits may be blocked (b5-b1 and d4-d1), but
     the king has time to take the other exit.

     \chessboard[
       style=puzzle,
       addfen={1c/c2c//4c/2+},
       mover=b,
       tinyboard,
       color=black,
       pgfstyle=straightmove,
       markmove=c1-a1,
       markmove=c1-e1,
     ]

   \end{sol}
\end{ex}

\Closesolutionfile{ans}

\section{King Escape puzzle answers}
\begin{Solution}{1}
     \chessboard[
       style=puzzle,
       addfen={//2+},
       mover=w,
       tinyboard,
       color=black,
       pgfstyle=straightmove,
       markmove=c3-a3,
       markmove=a3-a1
     ]

     It's just two moves. Since the board is symmetrical, there are
     essentially 4 equivalent ways to do the initial move. After
     you've broken the symmetry by moving once, there are two ways to
     end the game: a3-a1 or a3-5.
   
\end{Solution}
\begin{Solution}{2}
     \chessboard[
       style=puzzle,
       addfen={//2+},
       mover=w,
       tinyboard,
       color=black,
       markstyle=color,
       markfields={a3,c1,c5,e3},
       pgfstyle=straightmove,
       markmove=c3-b3,
       markmove=b3-b1,
       markmove=b1-a1,
     ]

     The blocked squares add a move to the kings escape
     plans. Attackers won't let your king just leave, so it's common
     for the king to make a winding path to the exit.
   
\end{Solution}
\begin{Solution}{3}
     \chessboard[
       style=puzzle,
       addfen={/1CC/2+C/2C/},
       mover=w,
       tinyboard,
       color=black,
       markstyle=color,
       markfields={a3,c1,c5,e3},
       pgfstyle=straightmove,
       markmove=b3-b4,
       markmove=c3-b3,
       markmove=b3-b1,
       markmove=b1-a1,
     ]

     Notice that your own pieces get in the king's way. Moving the
     guard on b3 slows the king down by one move, so it takes 4 in
     total.  There are certainly times when you'll wish you have just
     one fewer pieces on the board.
   
\end{Solution}
\begin{Solution}{4}
     Yes, unless captures are allowed.

     \chessboard[
       style=puzzle,
       addfen={2c//c+2c//1c},
       showmover=false,
       tinyboard,
       color=black,
       pgfstyle=straightmove,
       markmove=c3-b3,
       markmove=c1-b1,
     ]
     \chessboard[
       style=puzzle,
       addfen={2c//c+2c//1c},
       showmover=false,
       tinyboard,
       color=black,
       pgfstyle=straightmove,
       markmove=b3-b5,
       markmove=b5-a5,
     ]


     If it is possible to capture the king against the edge of the
     board and the exit (as is the case with many Tafl games), the
     king will be captured by 2\dots~b1-b4xb5.

     \chessboard[
       style=puzzle,
       addfen={1+c/1c/c3c/},
       showmover=false,
       tinyboard,
       color=black,
       pgfstyle=straightmove,
       markmove=b1-b4,
     ]

     The attackers can continually lay that trap by carefully chasing
     the king around the board.

     \chessboard[
       style=puzzle,
       addfen={2c//c+2c//1c},
       showmover=false,
       tinyboard,
       color=black,
       pgfstyle=straightmove,
       markmove=c3-b3,
       markmove=c1-b1,
     ]
     \chessboard[
       style=puzzle,
       addfen={2c//c/1+2c/1c},
       showmover=false,
       tinyboard,
       color=black,
       pgfstyle=straightmove,
       markmove=b3-b2,
       markmove=e3-e2,
     ]
     \chessboard[
       style=puzzle,
       addfen={2c//c/3+c/3c},
       showmover=false,
       tinyboard,
       color=black,
       pgfstyle=straightmove,
       markmove=b2-d2,
       markmove=b1-d1,
     ]
     \chessboard[
       style=puzzle,
       addfen={2c/3+c/c//3c},
       showmover=false,
       tinyboard,
       color=black,
       pgfstyle=straightmove,
       markmove=d2-d4,
       markmove=e2-e4,
     ]


   
\end{Solution}
\begin{Solution}{5}
     According to Linnaeus, when a king has a path to escape, the
     defender must say \emph{raichi}. In this case, moving to a3, c5
     or e3 would give the king one way to escape. But each of those
     paths can be blocked by an attacker (e2-a2, d4-d5 or d4-e4). But
     if the king moves to c1 the defender would need to say
     \emph{tuichu}, which means he has two exits and has won the
     game. Both of those exits may be blocked (b5-b1 and d4-d1), but
     the king has time to take the other exit.

     \chessboard[
       style=puzzle,
       addfen={1c/c2c//4c/2+},
       mover=b,
       tinyboard,
       color=black,
       pgfstyle=straightmove,
       markmove=c1-a1,
       markmove=c1-e1,
     ]

   
\end{Solution}



\section{An annotated game}\label{firstwin}

This is my first Talbut win on Board Game Arena, the online game
service. It's not a particularly well-played game, to be
honest. Reviewing the game, I've noticed several obvious mistakes on
both sides. 

\chessboard[
  style=tablut,
  addfen={3ccc/4c/2C1C/c3C3c/cc1C+CCcc/c3C3c/4C/4c/3ccc},
  mover=w,
  tinyboard,
  color=black,
  pgfstyle=straightmove,
  markmove=c5-c7
]
\chessboard[
  style=tablut,
  addfen={4cc/4c/2CcC/c3C3c/cc1C+CCcc/c3C3c/4C/4c/3ccc},
  mover=b,
  tinyboard,
  color=black,
  pgfstyle=straightmove,
  markmove=d9-d7
]
1. c5-c7 d9-d7

The Swedes (white) must clear out a path for their king to get to one
of the four corners. I don't know if my initial move is ideal, but my
goal was to start clearing out a space for the king. The outer guards
have more freedom to block the Muscovites (black) so that the inner
guards to make a hole for the king to leave the throne. To me the
Muscovite move is perhaps too aggressive. It does threaten two Swedish
guards and I'm certain to lose at least one if they want to take it.

\chessboard[
  style=tablut,
  addfen={4cc/4c/2CcC/c3C3c/cc1C+CCcc/c3C3c/3C/4c/3ccc},
  mover=w,
  tinyboard,
  color=black,
  pgfstyle=straightmove,
  markmove=e3-d3
]
\chessboard[
  style=tablut,
  addfen={ 4cc/4c/1c1cC/c3C3c/c2C+CCcc/c3C13c/3C/4c/3ccc},
  mover=b,
  tinyboard,
  color=black,
  pgfstyle=straightmove,
  markmove=b5-b7
]
2. e3-d3 b5-b7xc7

Instead of moving one of my threatened guards out of the way, I take
the initiative on the other side of the board. The Muscovites go ahead
and take one of the Swedish guards.

\chessboard[
  style=tablut,
  addfen={4cc/4c/1c1cC/c3C3c/c2C+C1cc/c3C1C1c/3C/4c/3ccc},
  mover=w,
  tinyboard,
  color=black,
  pgfstyle=straightmove,
  markmove=g5-g4
]
\chessboard[
  style=tablut,
  addfen={4c/4c/1c1c1c/c3C3c/c2C+C1cc/c3C1C1c/3C/4c/3ccc},
  mover=b,
  tinyboard,
  color=black,
  pgfstyle=straightmove,
  markmove=f9-f7
]
3. g5-g4 f9-f7xe7

Having already lost one guard, I proceed to sacrifice the other one
and concentrate on developing the opposite corner. That sacrifice was
accepted.

\chessboard[
  style=tablut,
  addfen={4c/4c/1c1c1c/c3C3c/c2C+C1cc/c2C2C1c/3C/4c/3ccc},
  mover=w,
  tinyboard,
  color=black,
  pgfstyle=straightmove,
  markmove=e4-d4
]
\chessboard[
  style=tablut,
  addfen={4c/4c/1c1c1c2c/c3C/c2C+C1cc/c2C2C1c/3C/4c/3ccc},
  mover=b,
  tinyboard,
  color=black,
  pgfstyle=straightmove,
  markmove=i6-i7
]
4. e4-d4 i6-i7?

I'm ready to move an inner guard in preparation of moving my king
toward one of the bottom corners. In particular, i1 is looking
particularly promising.

Meanwhile the Muscovites have yet to move any piece below the 5
line. Assuming they insist on moving i6, a better move would have been
4\dots~i6-g6 which more directly protects i9. But something more
direct to hemming in the king (4\dots~f1-f3, perhaps) would have been
better.

\chessboard[
  style=tablut,
  addfen={4c/4c/1c1c1c2c/c3C/c2C1Cc1c/c2C2C1c/3C+/4c/3ccc},
  mover=w,
  tinyboard,
  color=black,
  pgfstyle=straightmove,
  markmove=e5-e3
]
\chessboard[
  style=tablut,
  addfen={4c/4c/1c1c1c2c/c3C/c2C1Cc1c/c2C2C1c/3C+c/4c/3cc},
  mover=b,
  tinyboard,
  color=black,
  pgfstyle=straightmove,
  markmove=f1-f3
]
5. e5-e3 f1-f3

It's always momentous when the king steps off the throne. On the 3
row, he has easy access to the open h column with the possibility of
moving to the open c column if blocked. Indeed, 5\dots~f1-f3
effectively limits his movement toward the right side of the board.

\chessboard[
  style=tablut,
  addfen={4c/4c/1c1c1c2c/c3C/c2C1Cc1c/c2C2C1c/C3+c/4c/3cc},
  mover=w,
  tinyboard,
  color=black,
  pgfstyle=straightmove,
  markmove=d3-a3
]
\chessboard[
  style=tablut,
  addfen={4c/4c/1c1c1c2c/c3C/c2C1Cc1c/c2C2C1c/4+c/c/3cc},
  mover=b,
  tinyboard,
  color=black,
  pgfstyle=straightmove,
  markmove=e2-a2
]
6. d3-a3? e2-a2xa3??

My goal was to escape along the open c column and my plan was to use
the 3 row. However, I didn't consider 6\dots~d1-d3, which would have
blocked the king from the bottom half of the board. A better move
would have been 6.~d3-d2, which would have opened up the c column
without allowing the attackers to block.

Fortunately for me, the temptation to capture a3 proved too great. In
addition to failing to block my escape, the Muscovites gave me tempo
and the 2 row. From here on the only question is whether the king will
choose the right path.

\chessboard[
  style=tablut,
  addfen={4c/4c/1c1c1c2c/c3C/c2C1Cc1c/c2C2C1c/5c/c3+/3cc},
  mover=w,
  tinyboard,
  color=black,
  pgfstyle=straightmove,
  markmove=e3-e2
]
\chessboard[
  style=tablut,
  addfen={4c/4c/1c1c1c2c/c3C/c2C1Cc1c/c2C2C/5c/c3+3c/3cc},
  mover=b,
  tinyboard,
  color=black,
  pgfstyle=straightmove,
  markmove=i4-i2
]
7. e3-e2! i4-i2

At first glance 7.~e3-c3 gives the king an easy escape to either a1 or
a9 depending on the attacker's response. But consider 7\dots~e8-c8
8.~c3-c1 b7-b1. The king would be in a very tight situation and must
hope to get back to the h column somehow.

Moving down to the 2 row forces 7\dots~i4-i2 because of 8.~e2-i2 \dots
9.~i2-i1.

\chessboard[
  style=tablut,
  addfen={4c/4c/1c1c1c2c/c3C/c2C1Cc1c/c2C2C/5c/c6+c/3cc},
  mover=w,
  tinyboard,
  color=black,
  pgfstyle=straightmove,
  markmove=e2-h2
]
8. e2-h2 Muscovites resign

There are only two responses to the king moving to the h column:
\begin{itemize}
\item 8\dots~e1-h1 (Or maybe 8\dots~e1-h1 and hope the king will walk into the 9.~h2-h1 i2-h2 trap?)
\item 8\dots~i5-h5
\end{itemize}  
Whichever route the attackers block, the king will use the other.

\chessboard[
  style=brandubh,
  addfen={3c/3c/3C/ccC+Cccc/3C/3c/3c},
  mover=b,
]


\chessboard[
  style=tablut,
  addfen={3ccc/4c/4C/c3C3c/ccCC+CCcc/c3C/4C/4c3c/3ccc},
  mover=b,
  tinyboard,
  color=black,
  pgfstyle=straightmove,
  markmove=i4-i2
]
\chessboard[
  style=tablut,
  addfen={3ccc/4c/4C/c3C3c/ccCC+CCcc/c3C/7C/4c3c/3ccc},
  mover=w,
  tinyboard,
  color=black,
  pgfstyle=straightmove,
  markmove=e3-h3
]
1. i4-i2 e3-h3

\chessboard[
  style=tablut,
  addfen={3ccc/4c/4C/c3C3c/ccCC+CCcc/c3C/4c2C/8c/3ccc},
  mover=b,
  tinyboard,
  color=black,
  pgfstyle=straightmove,
  markmove=e2-e3
]
\chessboard[
  style=tablut,
  addfen={3ccc/4c/4C/c3C3c/cc1C+CCcc/c3C/4c2C/2C5c/3ccc},
  mover=w,
  tinyboard,
  color=black,
  pgfstyle=straightmove,
  markmove=c5-c2
]
2. e2-e3 c5-c2

\chessboard[
  style=tablut,
  addfen={3ccc/4c/4C/c3C3c/c1cC+CCcc/c3C/4c2C/2C5c/3ccc},
  mover=b,
  tinyboard,
  color=black,
  pgfstyle=straightmove,
  markmove=b5-c5
]
\chessboard[
  style=tablut,
  addfen={3ccc/4c/4C/c3C3c/c1cC+CCcc/c3C/4c2C/2C5c/3ccc},
  mover=w,
  tinyboard,
  color=black,
  pgfstyle=straightmove,
  markmove=c2-e2
]
3. b5-c5 c2-e2xe3

\chessboard[
  style=tablut,
  addfen={3ccc/4c/4C/c3C3c/c1cC+CCcc/c3C3c/4C2c/2C5c/3ccc},
  mover=w,
  tinyboard,
  color=black,
  pgfstyle=straightmove,
  markmove=c2-e2
]
\chessboard[
  style=tablut,
  addfen={3ccc/4c/4C/c3C3c/c1cC+CCcc/c3C3c/7C/4C/3ccc},
  mover=b,
  tinyboard,
  color=black,
  pgfstyle=straightmove,
  markmove=d1-d2
]
3. c2-e2xe3 d1-d2

4. h3-i3xi2 f1-f2xe2

5. g5-g3 e1-e3

6. e6-c6 i6-e6xe7

7. f5-f6 h5-g5

8. d5-d6xe6 a4-d4

9. e5-e7 f2-f4xe4

10. e7-i7 e8-i2

11. i7-h7 f9-h9

12. h7-h1

\clearpage

% \listofdiagrams


\url{http://aagenielsen.dk/hnefatafl_online.php}
\url{http://tafl.cyningstan.com/page/29/forming-a-strategic-plan}
\url{https://bonaludo.com/2019/02/06/tablut-and-linnaeus-contribution-to-the-world-of-games/}
\url{https://www.gutenberg.org/ebooks/34779}
\url{http://tafl.cyningstan.com/}
\url{https://www.fotevikensmuseum.se/d/en/vikingar/hur/spel/regler}
\url{https://boardgamegeek.com/geeklist/35348/how-tafl-articles-my-review-series-tafl-games}
\url{http://tafl.cyningstan.com/page/92/archaeological-finds}

\end{document}

% LocalWords:  Tafl Tablut Brandubh Muscovites
