\begin{Solution}{1}
     \chessboard[
       style=puzzle,
       addfen={//2+},
       mover=w,
       tinyboard,
       color=black,
       pgfstyle=straightmove,
       markmove=c3-a3,
       markmove=a3-a1
     ]

     It's just two moves. Since the board is symmetrical, there are
     essentially 4 equivalent ways to do the initial move. After
     you've broken the symmetry by moving once, there are two ways to
     end the game: a3-a1 or a3-5.
   
\end{Solution}
\begin{Solution}{2}
     \chessboard[
       style=puzzle,
       addfen={//2+},
       mover=w,
       tinyboard,
       color=black,
       markstyle=color,
       markfields={a3,c1,c5,e3},
       pgfstyle=straightmove,
       markmove=c3-b3,
       markmove=b3-b1,
       markmove=b1-a1,
     ]

     The blocked squares add a move to the kings escape
     plans. Attackers won't let your king just leave, so it's common
     for the king to make a winding path to the exit.
   
\end{Solution}
\begin{Solution}{3}
     \chessboard[
       style=puzzle,
       addfen={/1CC/2+C/2C/},
       mover=w,
       tinyboard,
       color=black,
       markstyle=color,
       markfields={a3,c1,c5,e3},
       pgfstyle=straightmove,
       markmove=b3-b4,
       markmove=c3-b3,
       markmove=b3-b1,
       markmove=b1-a1,
     ]

     Notice that your own pieces get in the king's way. Moving the
     guard on b3 slows the king down by one move, so it takes 4 in
     total.  There are certainly times when you'll wish you have just
     one fewer pieces on the board.
   
\end{Solution}
\begin{Solution}{4}
     Yes, unless captures are allowed.

     \chessboard[
       style=puzzle,
       addfen={2c//c+2c//1c},
       showmover=false,
       tinyboard,
       color=black,
       pgfstyle=straightmove,
       markmove=c3-b3,
       markmove=c1-b1,
     ]
     \chessboard[
       style=puzzle,
       addfen={2c//c+2c//1c},
       showmover=false,
       tinyboard,
       color=black,
       pgfstyle=straightmove,
       markmove=b3-b5,
       markmove=b5-a5,
     ]


     If it is possible to capture the king against the edge of the
     board and the exit (as is the case with many Tafl games), the
     king will be captured by 2\dots~b1-b4xb5.

     \chessboard[
       style=puzzle,
       addfen={1+c/1c/c3c/},
       showmover=false,
       tinyboard,
       color=black,
       pgfstyle=straightmove,
       markmove=b1-b4,
     ]

     The attackers can continually lay that trap by carefully chasing
     the king around the board.

     \chessboard[
       style=puzzle,
       addfen={2c//c+2c//1c},
       showmover=false,
       tinyboard,
       color=black,
       pgfstyle=straightmove,
       markmove=c3-b3,
       markmove=c1-b1,
     ]
     \chessboard[
       style=puzzle,
       addfen={2c//c/1+2c/1c},
       showmover=false,
       tinyboard,
       color=black,
       pgfstyle=straightmove,
       markmove=b3-b2,
       markmove=e3-e2,
     ]
     \chessboard[
       style=puzzle,
       addfen={2c//c/3+c/3c},
       showmover=false,
       tinyboard,
       color=black,
       pgfstyle=straightmove,
       markmove=b2-d2,
       markmove=b1-d1,
     ]
     \chessboard[
       style=puzzle,
       addfen={2c/3+c/c//3c},
       showmover=false,
       tinyboard,
       color=black,
       pgfstyle=straightmove,
       markmove=d2-d4,
       markmove=e2-e4,
     ]


   
\end{Solution}
\begin{Solution}{5}
     According to Linnaeus, when a king has a path to escape, the
     defender must say \emph{raichi}. In this case, moving to a3, c5
     or e3 would give the king one way to escape. But each of those
     paths can be blocked by an attacker (e2-a2, d4-d5 or d4-e4). But
     if the king moves to c1 the defender would need to say
     \emph{tuichu}, which means he has two exits and has won the
     game. Both of those exits may be blocked (b5-b1 and d4-d1), but
     the king has time to take the other exit.

     \chessboard[
       style=puzzle,
       addfen={1c/c2c//4c/2+},
       mover=b,
       tinyboard,
       color=black,
       pgfstyle=straightmove,
       markmove=c1-a1,
       markmove=c1-e1,
     ]

   
\end{Solution}
